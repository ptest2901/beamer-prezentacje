\documentclass[aspectratio=169]{beamer}

\usepackage{polski}
\usetheme[lang=pl,hr=true,vertical=true]{NewPwr}

\newcommand{\RepoURL}{https://twoj-link-do-repozytorium.example}

\title{Sztuczna inteligencja w praktyce}
\subtitle{Krótkie wprowadzenie i zastosowania}
\author{Student PWr}
\institute{Politechnika Wrocławska}
\date{\today}

\begin{document}

\begin{frame}
\maketitle
\end{frame}

\begin{frame}{Zaliczenie zadania}
\textbf{Repozytorium ze źródłami prezentacji:}\\[0.3em]
\href{\RepoURL}{\RepoURL}\\[1em]
\begin{itemize}
\item pliki \texttt{.tex}, grafiki (jeśli użyte)
\item bez plików pomocniczych generowanych przy kompilacji PDF
\end{itemize}
\end{frame}

\begin{frame}{Informacje techniczne}
\begin{itemize}
\item Pakiet: \texttt{beamer}
\item Styl: \texttt{NewPwr} (\texttt{hr=true}, \texttt{vertical=true})
\item Format slajdów: 16:9
\end{itemize}
\end{frame}

\begin{frame}{Plan prezentacji}
\tableofcontents
\end{frame}

\section{Podstawy AI}

\begin{frame}{Czym jest sztuczna inteligencja?}
\begin{itemize}
\item AI to metody, które pozwalają systemom analizować dane i wspierać decyzje.
\item Najczęściej spotykane podejście: uczenie maszynowe (ML).
\item Przykłady: rozpoznawanie obrazów, analiza tekstu, przewidywanie zdarzeń.
\end{itemize}
\end{frame}

\begin{frame}{Uczenie maszynowe w skrócie}
\begin{itemize}
\item Nadzorowane: dane + etykiety.
\item Nienadzorowane: szukanie wzorców.
\item Ze wzmocnieniem: wybór akcji w środowisku.
\end{itemize}
\end{frame}

\section{Modele generatywne}

\begin{frame}{Modele generatywne i LLM}
\begin{itemize}
\item Modele generatywne tworzą nowe treści, np. tekst.
\item LLM generuje odpowiedź na podstawie kontekstu.
\item Zastosowania: streszczenia, szkice, wsparcie analizy dokumentów.
\end{itemize}
\end{frame}

\begin{frame}{Zasady bezpiecznego użycia}
\begin{itemize}
\item Nie wklejaj danych poufnych do narzędzi zewnętrznych.
\item Wymagaj konkretnego formatu odpowiedzi (lista, kroki, tabela).
\item Zawsze weryfikuj wynik przed użyciem.
\end{itemize}
\end{frame}

\section{Zastosowania}

\begin{frame}{Zastosowania AI w przemyśle}
\begin{itemize}
\item Kontrola jakości i detekcja defektów.
\item Predykcyjne utrzymanie ruchu na podstawie danych z czujników.
\item Prognozowanie i optymalizacja planowania.
\item Analiza dokumentów technicznych i raportów.
\end{itemize}
\end{frame}

\begin{frame}{Ryzyka i ograniczenia}
\begin{itemize}
\item Halucynacje: odpowiedź może brzmieć pewnie, ale być błędna.
\item Jakość danych ma kluczowe znaczenie dla jakości wyników.
\item Odpowiedzialność: decyzje krytyczne podejmuje człowiek.
\end{itemize}
\end{frame}

\begin{frame}{Podsumowanie}
\begin{itemize}
\item AI wspiera analizę i automatyzuje powtarzalne zadania.
\item Modele generatywne są przydatne, ale wymagają kontroli.
\item Najważniejsze: cel, dane, bezpieczeństwo, weryfikacja.
\end{itemize}
\end{frame}

\begin{frame}
\centering
\vfill
{\LARGE Dziękuję za uwagę!}\\[0.6em]
Pytania?
\vfill
\end{frame}

\begin{frame}{Materiały (skrócone)}
\begin{itemize}
\item I. Goodfellow, Y. Bengio, A. Courville, \emph{Deep Learning}, 2016
\item A. Vaswani i in., \emph{Attention Is All You Need}, 2017
\item Dokumentacja klasy \texttt{beamer}
\end{itemize}
\end{frame}

\end{document}
